\documentclass[a4paper, 10pt, notitlepage]{article}

\usepackage{moreverb} %para importar codigo

\usepackage{pepotina} %paquete personal para la caratula del DC

\usepackage[spanish,activeacute]{babel}
\usepackage{babel} %paquete de idioma

\usepackage[latin1]{inputenc}

\usepackage{color}

\usepackage{caeycaeING}

\usepackage{fancyhdr} %linea sup con comentarios

%\usepackage{listings}

\usepackage{lscape} %para hoja apaisada

\usepackage{framed} %para crear cajas de texto

\usepackage{lastpage} %ultima pagina

\usepackage{pstricks}
\usepackage{uml} %UML


\addtolength{\topmargin}{-50pt} 
\addtolength{\textwidth}{105pt}
\addtolength{\textheight}{120pt}
\addtolength{\oddsidemargin}{-50pt}

%\newcommand{\minix}{\textsl{minix }}

%%% Encabezado y pie de p'agina
\pagestyle{fancy}
\fancyhead[LO]{Ingenieria del Software I}
\fancyhead[C]{}
\fancyhead[RO]{P\'agina \thepage\ de \pageref{LastPage}}
\renewcommand{\headrulewidth}{0.4pt}
\fancyfoot{}

\newcommand{\falta}{ \begin{framed}	\begin{center} \hspace{1cm} \Large FALTA \normalsize \hspace{1cm} \end{center} \end{framed}}

\begin{document}

\universidad{Universidad de Buenos Aires}
\facultad{Facultad de ciencias exactas y naturales}
\departamento{Departamento de Computacion}
\materia{Ingenieria del Software I}
\resumen{En este trabajo...}
\keys{UML, Objetivos}
\titulo{Trabajo Practico}
\subtitulo{Grupo N� 2}
\fecha{1er Cuatrimeste 2008}
\footspace{1cm}
\integrante{Aquino, Isis}{caeycae@gmail.com}
\integrante{Alvarez, Maria}{caeycae@gmail.com}
\integrante{Garcia, Ana Daniela}{caeycae@gmail.com}
\integrante{Engler, Christian Alejandro}{caeycae@gmail.com}

%caratula
%\maketitle{} %OK

%\tableofcontents
%\newpage

\newcommand{\pj}{{\bf posible jugador }}

\begin{cu}{Comprando fichas}
	\descripxizq=1cm
		\defln{Este caso de uso explica como el \pj interactua con el sistema con el fin de comprar fichas}
		\actor{\pj}
		%\actorsec{}
		\pre{true}
		\post{el \pj dispone de las fichas que quiso comprar}
	\begin{cursoe}{}
	
		\paso{1 El \pj ingresa al (sistema) pagina web del casino}
		{}
		
		\paso{2 El \pj ingresa su nombre y numero de documento}
		{}
		
		\paso{3 \pj ingresa el numero de su cuenta corriente}
		{2.1 El sistema rechaza al \pj por ser menor de edad. Ir a fin de caso de uso}
		
		\paso{4 \pj ingresa el numero de su cuenta corriente}
		{}
		
		\paso{5 El sistema verificar� la cuanta corriente ingresada. \incl{Verificando cuenta} }
		{}
		
		\paso{6 Si el \pj deseea jugar podra seleccionar el juego que desee \exti{Jugando juegos} }
		{5.1 En caso de que la verificacion sea erronea ir a fin de caso de uso}
		
		\paso{7 Fin de caso de uso}{}
		
	\end{cursoe}
	
	\paragraph{Preguntas:}
		\begin{flushleft}
		�Debe ser mayor de ciarta edad para jugar?\\
		�Ir a Vendiendo fichas? \\
		�Extiende o Incluye caso de uso Vendiendo fichas?
		\end{flushleft}
	
\end{cu}


\newpage


\newcommand{\jj}{{\bf jugador de juegos }}

\begin{cu}{Jugando juegos}
	\descripxizq=1cm
		\defln{Este caso explica como un jugador de juegos selecciona el juego que desea jugar}
		\actor{\jj}
		\pre{true}
		\post{el \jj selecciona el juego que desea jugar}
	\begin{cursoe}{}
			
		\paso{1 Fin de caso de uso}{}
		
	\end{cursoe}
	
	\paragraph{Preguntas:}
		\begin{flushleft}
		�precondicion true?\\
		�Debe seleccionar algun juego en particular o la cancelacion en una excepcion?
		\end{flushleft}
	
\end{cu}











%\begin{cu}{Vendiendo pasajes}
%\defln{Este caso de uso expilica la interaccion 
%del vendedor con nuestro sistemas de ventas}
%\actor{vendedor de pasajes}
%\actorsec{vendedor de pasajes secundario}
%\actorsec{sistema online}
%\pre{precondicion}
%\post{postcondicion}
%\begin{cursoe}{}
%\paso{a}{b}
%\end{cursoe}
%\end{cu}
%
%De esta manera escribiremos en caso de uso:
%
%\begin{verbatim}
%\begin{cu}{Vendiendo pasajes}
%\defln{Este caso de uso expilica la interaccion 
%del vendedor con nuestro sistemas de ventas}
%\actor{vendedor de pasajes}
%\actorsec{vendedor de pasajes secundario}
%\actorsec{sistema online}
%\pre{precondicion}
%\post{postcondicion}
%\begin{cursoe}{}
%\paso{a}{b}
%\end{cursoe}
%\end{cu}
%\end{verbatim}

\end{document}



