\documentclass[a4paper, 10pt, notitlepage]{article}
 
\usepackage[pdftex]{graphicx}
\usepackage{moreverb} %para importar codigo
%\usepackage{pepotina} %paquete personal para la caratula del DC
\usepackage[spanish,activeacute]{babel}
\usepackage{babel} %paquete de idioma
\usepackage[latin1]{inputenc}
%\usepackage{color}
\usepackage{hyperref}
%\usepackage{caeycaeING}
\usepackage{fancyhdr} %linea sup con comentarios
\usepackage{lscape} %para hoja apaisada
\usepackage{framed} %para crear cajas de texto
\usepackage{lastpage} %ultima pagina

\usepackage{listings}
%\lstset{
%  breaklines=true,          % line wrapping on
%  language=ocl,
%  frame=ltrb,
%  framesep=5pt,
%  basicstyle=\normalsize,
%  keywordstyle=\ttfamily\color{OliveGreen},
%  identifierstyle=\ttfamily\color{CadetBlue}\bfseries,
%  commentstyle=\color{Brown},
%  stringstyle=\ttfamily,
%  showstringspaces=ture
%}
 
\newcommand{\HRule}{\rule{\linewidth}{0.5mm}}

\addtolength{\topmargin}{-50pt} 
\addtolength{\textwidth}{105pt}
\addtolength{\textheight}{120pt}
\addtolength{\oddsidemargin}{-50pt}

\usepackage{fancyhdr} %linea sup con comentarios
\pagestyle{fancy}
\fancyhead[LO]{Ingenieria del Software I}
\fancyhead[C]{}
\fancyhead[RO]{P\'agina \thepage\ de \pageref{LastPage}}
\renewcommand{\headrulewidth}{0.4pt}
\fancyfoot{}

 
\begin{document}

%\setcounter{section}{1} 
 
\begin{titlepage}
 
\begin{center}
 
%% \universidad{Universidad de Buenos Aires}
%%\facultad{Facultad de ciencias exactas y naturales}
%%\departamento{Departamento de Computacion}
%%\materia{Ingenieria del Software I}
%%\resumen{Proyecto casino online}
%%\keys{UML, Objetivos, Agentes, Casos De Uso, Diagrama De Contexto, Modelo Conceptual, OCL, Diagrama de Actividades, FSM}
%%\titulo{Proyecto: Casino Online}
%%\subtitulo{Informe 1: Analisis de Requerimientos y especificaci�n}
%%\grupo{Numero de grupo: 2}
%%\fecha{1er Cuatrimeste 2008}
%%\footspace{1cm}
%%\integrante{Aquino, Isis}{313/05}{isisaquino@yahoo.com.ar}
%%\integrante{Alvarez, Maria de los Angeles}{264/05}{mdelosaalvarez@hotmail.com}
%%\integrante{Engler, Christian Alejandro}{314/05}{caeycae@gmail.com}

 
 
 
 
 
% Upper part of the page
\includegraphics[width=0.5\textwidth]{./logo_uba_p.jpg}\\[1cm]
 
\textsc{\LARGE Universidad de Buenos Aires}\\
\textsc{\LARGE Facultad de ciencias exactas y naturales}\\[1.5cm]
 
\textsc{\Large Departamento de Computacion}\\
\textsc{\Large Ingenieria del Software I}\\[0.5cm]
 
% Title
\HRule \\[0.4cm]
{ \huge \bfseries Proyecto Casino Online}\\[0.4cm]
 
\HRule \\[1.5cm]
 
\begin{center}
	\begin{flushleft} \large
	\emph{Author:}\\
		Isis \textsc{Aquino} $313/05$ \\
		Maria de los Angeles \textsc{Alvarez} $264/05$\\
		Christian Alejandro \textsc{Engler} $314/05$
	\end{flushleft}
\end{center}.\\[2cm]

\begin{center}
	\begin{flushleft} \large
	\emph{Supervisor:} \\
		Sergio \textsc{D'Arrigo}
	\end{flushleft}
\end{center}
 
\vfill
 
% Bottom of the page
{\large \today}
 
\end{center}
 
\end{titlepage}


\tableofcontents

\newpage

\section{Introduccion}

\section{Cambios con respecto al dise�o}

\section{Testing}

\subsection{Abrir Casino}
\subsubsection{Elecciones}
\subsubsection{Resultados}
\subsubsection{Casos de Test}

\subsection{Tirar Craps}
\subsubsection{Elecciones}
\subsubsection{Resultados}
\subsubsection{Casos de Test}

\section{Organizacion del Codigo}
\label{sec:OrganizacionDelCodigo}

El la carpeta codigo se encuentra todo el codigo y archivos necesarios para la ejecucion y compilacion de dicho codigo.


\begin{itemize}
	\item /doc - documentacion autogenerada (no  completa)
	\item /etc - Archivos necesarios para ciertos chequeos de codigo
	\item /jar - codigo compilado
	\item /lib - librerias necesarias para la compilacion
	\item /report - varios reporte de calidad de codigo
	\item /src - codigo java
		\subitem /casino - paquete casino
		\subitem /configuracion - archivos de configuracion
			\subsubitem CFGSaldo.xml - configuracion del saldo de los pozos
			\subsubitem fichasValidas.xml - configuracion de fichas validas
			\subsubitem generalConfig.properties - configuracion de minimo pozo feliz y descuento todos ponen
			\subsubitem listaJugadores.xml - lista de jugadores y sus saldos
			\subsubitem mensajeroConfig.properties - filtros de los mensajeros y mailbox
			\subsubitem modificaciones.csv - especifica para cada apuesta de craps cierta accion
			\subsubitem pagos.csv - especifica en que relacion y cuando se deben pagar las apuestas
		\subitem /core - paquete core
		\subitem /craps - paquete craps
		\subitem /interpretador - paquete interpretador
		\subitem /mensajero - paquete mensajero
		\subitem /observerCraps - paquete observador de craps
		\subitem /parser - paquete con los parsers del casino
		\subitem /server - paquete con el iniciador del casino
		\subitem /servicios - paquete servicios
		\subitem log4j.properties - archivo de configuracion de loggeo
	\item /xml - xml de ejemplo
	\item build.xml - archivo ant para la compilacion del proyecto
\end{itemize}




\end{document}


