
\subsection{Analisis de Objetivos}

Para lograr tener uns visi�n intencional del sistema, tanto en su parte funcional como en la no funcional, presentamos los objetivos que identificamos organizandolos en un Diagrama de Objetivos\footnote{Objetivos organizados en uno o mas arboles, donde tenemos que casa objetivos se logra gracias al cumplimiento de los objetivos dependientes (hijos inmediatos)}

\begin{figure}[h!]
	\centering
		\includegraphics[scale=0.25]{img/doCasino.png}
	\caption{Diagrama de Objetivos \label{fig:doCasino}}
\end{figure}

\begin{framed}

\depto Con este Diagrama de Objetivos intentamos dar el primer paso en la definicion del alcance del sistema y la justificacion de los requerimientos encontrados.
Tambien sera de utilidad para el analisis de los supuestos sobre los que se basar� el sistema.

\end{framed}

\subsection{Analisis de agentes involucrados}

Para comprender mejor la manera en que deberan relacionarse los distintos sujetos del mundo con el sistema a construir y sus interacciones, estucturamos el mundo con un Diagrama de contexto\footnote{Diagrama de contexto: Diagrama donde las cajas son agentes activos del sistema y las flechas son las interacciones basicas}

Este diagrama tambien nos da otra mirada del alcance y expectativas del software a construir.

\begin{figure}[h!]
	\centering
		\includegraphics[scale=0.38]{img/dcCasino.png}
	\caption{Diagrama de Contexto \label{fig:dcCasino}}
\end{figure}

\begin{framed}

\depto Todas las interacciones de los agentes con el sistema han sido asignadas a uno o mas Casos de Uso que explicaran de forma mas detallada dichas interacciones. De esta manera podemos garantizar que las interacciones mas relevantes del sistema seran correctamente detalladas

\end{framed}

\subsection{Lista de Requerimientos}

\begin{enumerate}
\item Mostrar en todo momento el monto de los posos
\item Generar informe electr�nico: Ranking de jugadores
\item Generar informe electronico: Estado Actual
\item Generar informe electronico: Destalle de movimientos por jugador
\item El sistema debe funcionar en red
\item Generar jugada feliz autom�ticamente
\item Generar jugada todos ponen autom�ticamente
\item El sistema deber� contar con un modo dirigido
\item El sistema deber� permitir generar jugada feliz manualmente
\item El sistema deber� permitir generar jugada todos ponen manualmente
\item Permitir la configuraci�n est�tica del monto m�nimo de los posos
\item Proveer jugo tragamonedas
\item Proveer juego craps
\item El juego tragamonedas debe contar con un premio gordo progresivo
\item No se debe permitir el solapamiento de jugadas todos ponen (�felices?)
\item No se debe permitir que un jugador juegue en mas de una mesa al mismo tiempo
\item Los clientes no podr�n apostar mas que el saldo permitido por el departamento de \item marketing (por el archivo prove�do por...)
\item Los clientes VIP podr�n apostar ilimitadamente
\item Los clientes podr�n abrir mesas
\item Los clientes podr�n unirse a mesas (s� el juego lo permite)
\item Las mesas vaci�s se cerraran autom�ticamente
\item El casino no se podr� cerrar mientras haya gente jugando
\item No debe haber limite de mesas abiertas
\item Se deber� modificar la el saldo del cliente en cada apuesta
\item Las apuestas se har�n por medio de fichas
\item Las fichas ser�n ilimitadas
\item Las pantallas mostraran la informaci�n necesaria para el desarrollo del juego
\item Las pantallas mostraran la informaci�n necesaria del estado del juego
\item Las pantallas mostraran el estado de la cuenta del jugador
\item El cliente podr� elegir entre varios valores de fichas de las maquinas tragamonedas, configurables por el administrador
\end{enumerate}

\begin{framed}
\textbf{TINTERO: } Nos ha quedado en el tintero relacionar en el imforme los requerimientos con las distintas secciones del trabajo.\\
De todas maneras utilizamos esta lista de el modelo de objetivos como base para evaluar completitud de este informe.\\
Tambien hemos realizado otros tipos de trazabilidad con otros modelos.
\end{framed}

\newpage
\subsection{Modelo Conceptual}

Identificamos y explicamos los conceptos funadamentales y las propiedades definitorias de ellos, y los estructuramos en un Modelo Conceptual\footnote{Casa caja representa una agrupacion de objetos reconocidos del dominio del problema que se caracterizan por tener propiedades similares, las lineas son asociaciones estre estos conceptos que indica alguna vinculacion significativa entre ellos}

\begin{center}
		\includegraphics[scale=0.38]{img/dclCasino.png}
\end{center}

Dado las limitaciones del Modelo Conceptual, agregaremos algunos invarientes para identificar mejor los estados deseables del sistema. Ver Anexo \ref{an2}


\subsection{Casos de Uso}

\begin{figure}[h!]
	\centering
		\includegraphics[scale=0.38]{img/casosDeUso.png}
	\caption{Casos de Uso \label{fig:cuCasino}}
\end{figure}

\begin{figure}[h!]
	\centering
		\includegraphics[scale=0.38]{img/herencia.png}
	\caption{Casos de Uso \label{fig:herencia}}
\end{figure}

%ACTORES
\newcommand{\pc}{{\bf Cliente del casino }}
\newcommand{\adm}{{\bf Administrador }}
%\newcommand{\ptra}{{\bf Potencial jugador de tragamonedas }}
\newcommand{\jutra}{{\bf Jugador de Tragamonedas }}
\newcommand{\pjc}{{\bf Potencial jugador de Craps }}
\newcommand{\jc}{{\bf Jugador de Craps }}
\newcommand{\jac}{{\bf Apostador de Craps }}
\newcommand{\emc}{{\bf Empleado Contable }}
\newcommand{\emk}{{\bf Empleado de Marketing }}
%CASOS DE USO
\newcommand{\ic}{{\bf Ingresando a casino }}
\newcommand{\salc}{{\bf Saliendo del casino }}
\newcommand{\atra}{{\bf Abriendo Mesa Tragamonedas }}
\newcommand{\jtra}{{\bf Jugando Tragamonedas }}
\newcommand{\actm}{{\bf Activando Modo Dirigido }}
\newcommand{\desm}{{\bf Desactivando Modo Dirigido }}
\newcommand{\jugf}{{\bf Jugada Feliz }}
\newcommand{\jugtp}{{\bf Jugada Todos Ponen }}
\newcommand{\ac}{{\bf Abriendo Casino }}
\newcommand{\cc}{{\bf Cerando Casino }}



%----------------------------------------------------------
% CASO DE USO INGRESANDO A CASINO
%----------------------------------------------------------
\begin{cu}{\ic}
	\descripxizq=1cm
		\defln{Este caso de uso explica como un \pc ingresa en el casino }
		\actor{\pc}
		\pre{El casino debe estar abierto y el jugador debe estar fuera del casino}
		\post{El \pc ha ingresado al casino }

	\begin{cursoe}{}
	
		\paso{1 El \pc ingresa su nombre de usuario}{}
		
		\paso{2 El sistema verifica si el nombre corresponde a un usuario valido}{}
		
		\paso{3 El sistema ingresa al \pc al casino}{}
		
		\paso{4 El sistema informa que el \pc ha ingresado al casino satisfactoriamente}{4.1 El sistema informa que el usuario es invalido. Ir a paso 5}
		
		\paso{5 Fin C.U.}{}	
		
	\end{cursoe}
	
\end{cu}

%----------------------------------------------------------
% CASO DE USO SALIENDO A CASINO
%----------------------------------------------------------

\begin{cu}{\salc}
	\descripxizq=1cm
		\defln{Este caso de uso explica como un \pc sale del casino }
		\actor{\pc}
		\pre{El casino debe estar abierto y el \pc debe estar dentro del casino}
		\post{El \pc ha salido del casino }

	\begin{cursoe}{}
		
		\paso{1 Si el \pc no esta jugando ir a paso 4}{}
		
		\paso{2 El sistema informa que de continuar, perder� todas las apuestas activas}{}
	
		\paso{3 El sistema cancela todas las apuestas del \pc, y lo retira de la mesa correspondiente}{3.1 El \pc cancela la operacion ir a paso 6}
		
		\paso{4 El sistema retira al \pc del casino}{}
		
		\paso{5 El sistema informa que el \pc ha salido del casino}{}
		
		\paso{6 Fin C.U.}{}
		
	\end{cursoe}
	
\end{cu}

%----------------------------------------------------------
% CASO DE USO SALIENDO A CASINO
%----------------------------------------------------------

\begin{cu}{\jtra}
	\descripxizq=1cm
		\defln{Este caso de uso explica como un \jutra que ha ingresado en el casino puede jugar y apostar en una mesa de tragamonedas}
		\actor{\jutra}
		\pre{El \jutra debe estar dentro del casino y no estar jugando en ninguna otra mesa}
		\post{El \jutra ha jugado al tragamonedas }

	\begin{cursoe}{}
	
		\paso{1 El \jutra selecciona el valor de la ficha de la nueva mesa de tragamonedas}{}
		
		\paso{2 El sistema crea una nueva mesa para el \jutra}{}
			
		\paso{3 El sistema informa que el jugador ha ingresado a una mesa de tragamonedas}{}
		
		\paso{4 El \jutra selecciona la cantidad de fichas a apostar (entre las permitidas) }{}
		
		\paso{5 Si el \jutra no es VIP el sistema verifica si tiene saldo suficiente para la apuesta}{}
			
		\paso{6 Si el saldo es sufiente el sistema valida la apuesta, sino el sistema informa que no tiene saldo suficiente para realizar esta apuesta. Ir a Fin C.U.}{}
				
		\paso{7 El sistema incrementa el pozo "Premio Gordo Progresivo" }{}
		
		\paso{8 El sistema informa si la jugada es normal, feliz o todos ponen}{}
		
	  \paso{9 El sistema da comienzo al juego }{}		
	  
	  \paso{10 El sistema muestra el resultado del juego }{}
	  
	  \paso{11 Si el resultado es ganador el sistema acredita el premio correspondiente }{}
	  
% hacer CU apost trag
		
		\paso{12 Fin C.U.}{}
		

	\end{cursoe}


%\paragraph{Comentario: }
%\begin{flushleft}
%\end{flushleft}
\end{cu}

