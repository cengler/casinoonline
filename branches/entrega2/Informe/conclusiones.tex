Para la conclusi�n de este segundo informe hemos decidido hacer un breve resumen de la metodolog�a que hemos utilizado para desarrollarlo, para demostrar que este es un trabajo no de avance lineal sino, un espiral donde constantemente se est� consultando los documentos base, es decir, toda la especificaci�n presentada en el Informe 1, y retocando el trabajo actual. \\
En primer lugar decidimos consultar nuestro modelo conceptual. Aunque nos sirvi� para refrescar las ideas, no dimos cuenta que no aplicaba mucho al dise�o pero s� para darnos una idea general. Empezamos a investigar un poco mas los principios de dise�o y encontramos algunos que nos fueron de gran utilidad.\\
Empezamos a usar interfaces en las partes que vimos convenientes, como en Casino, Jugador e Invitado, asi en un futuro podr�an ampliar sus funciones, como muestra el Dependency Inversion Principle. Y en algunos casos usamos interfaces de interfaces para poder controlar otras dependencias(Interface Segregation Principle), como es el caso de los seleccionadores de resultados del tragamonedas y del craps, ya que en un futuro pueden agregarse otras formas de seleccionar los resultados aparte del modo normal(por azar) y el dirigido.\\
Al momento de realizar el diagrama de secuencias volvimos a consultar el informe 1, especialmente la parte de los Casos de Uso y las FSM. De estos diagramas de secuencias surgieron la mayor parte de los m�todos.\\
Luego aparecio el protocolo y sus actualizaciones, por lo que vimos necesario modificar y retocar en varias oportunidades nuestro diagrama de clases y por lo tanto tambien el de secuencias.\\
Tambien estudiamos los distintos patrones. Un patr�n que nos fueron de utilidad fue el Singleton, para los Manejadores especialmente, ya que asi podr�an comunicarse entre si. Otro tipo de patrones (como el Facade para el seleccionador de resultado o de tipo de jugada) sencillamente aparecieron, no fue necesario forzar el dise�o.\\
En fin, el dise�o requirio de pensar a futuro, en el `como' lo vamos a implementar y los cambios que pudieran surgir, aprovechando las herramientas y principios provistos para lograrlo.
