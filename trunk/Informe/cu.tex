%ACTORES
\newcommand{\pc}{{\bf Cliente del casino }}
\newcommand{\adm}{{\bf Administrador }}
%\newcommand{\ptra}{{\bf Potencial jugador de tragamonedas }}
\newcommand{\jutra}{{\bf Jugador de Tragamonedas }}
\newcommand{\pjc}{{\bf Potencial jugador de Craps }}
\newcommand{\jc}{{\bf Jugador de Craps }}
\newcommand{\jac}{{\bf Apostador de Craps }}
\newcommand{\emc}{{\bf Empleado Contable }}
\newcommand{\emk}{{\bf Empleado de Marketing }}
\newcommand{\aptra}{{\bf Apostador de Tragamonedas }}
\newcommand{\jucr}{{\bf Jugador de Craps }}
%CASOS DE USO
\newcommand{\ic}{{\bf Ingresando a casino }}
\newcommand{\salc}{{\bf Saliendo del casino }}
\newcommand{\atra}{{\bf Abriendo Mesa Tragamonedas }}
\newcommand{\jtra}{{\bf Jugando Tragamonedas }}
\newcommand{\actm}{{\bf Activando Modo Dirigido }}
\newcommand{\desm}{{\bf Desactivando Modo Dirigido }}
\newcommand{\jugf}{{\bf Jugada Feliz }}
\newcommand{\jugtp}{{\bf Jugada Todos Ponen }}
\newcommand{\ac}{{\bf Abriendo Casino }}
\newcommand{\cc}{{\bf Cerando Casino }}
\newcommand{\apotra}{{\bf Apostando en Tragamonedas }}
\newcommand{\incr}{{\bf Ingresando a mesa de Craps }}
\newcommand{\scr}{{\bf Saliendo de mesa de Craps }}
\newcommand{\jcr}{{\bf Jugando Craps }}
\newcommand{\apcr}{{\bf Apostando Craps }}
\newcommand{\admin}{{\bf administrador}}
\newcommand{\ccas}{{\bf Cerrando Casino}}
\newcommand{\empmark}{{\bf Empleado de Marketing}}
\newcommand{\infomovjug}{{\bf informe Movimientos Por jugador}}
\newcommand{\infoestact}{{\bf informe Estado Actual}}
\newcommand{\inforankjug}{{\bf informe Ranking de Jugadores}}
\newcommand{\infdmj}{{\bf Pidiendo informe Detalles de movimientos por jugador}}
\newcommand{\infea}{{\bf Pidiendo informe Estado Actual}}
\newcommand{\infrj}{{\bf Pidiendo informe Ranking de Jugadores}}


\newpage
%----------------------------------------------------------
% CASO DE USO INGRESANDO A CASINO
%----------------------------------------------------------
\label{lic}

\begin{cu}{\ic}
	\descripxizq=1cm
		\defln{Este caso de uso explica como un \pc ingresa en el casino }
		\actor{\pc}
		\pre{El casino debe estar abierto y el jugador debe estar fuera del casino}
		\post{El \pc ha ingresado al casino }

	\begin{cursoe}{}
	
		\paso{1 El \pc ingresa su nombre de usuario}{}
		
		\paso{2 El sistema verifica si el nombre corresponde a un usuario valido}{}
		
		\paso{3 El sistema ingresa al \pc al casino}{}
		
		\paso{4 El sistema informa que el \pc ha ingresado al casino satisfactoriamente}{4.1 El sistema informa que el usuario es invalido. Ir a paso 5}
		
		\paso{5 Fin C.U.}{}	
		
	\end{cursoe}
	
\end{cu}

%----------------------------------------------------------
% CASO DE USO SALIENDO DEL CASINO
%----------------------------------------------------------
\label{lsalc}

\begin{cu}{\salc}
	\descripxizq=1cm
		\defln{Este caso de uso explica como un \pc sale del casino }
		\actor{\pc}
		\pre{El casino debe estar abierto y el \pc debe estar dentro del casino}
		\post{El \pc ha salido del casino }

	\begin{cursoe}{}
		
		\paso{1 Si el \pc no esta jugando ir a paso 4}{}
		
		\paso{2 El sistema informa que de continuar, perder� todas las apuestas activas}{}
	
		\paso{3 El sistema cancela todas las apuestas del \pc, y lo retira de la mesa correspondiente}{3.1 El \pc cancela la operacion ir a paso 6}
		
		\paso{4 El sistema retira al \pc del casino}{}
		
		\paso{5 El sistema informa que el \pc ha salido del casino}{}
		
		\paso{6 Fin C.U.}{}
		
	\end{cursoe}
	
\end{cu}

\newpage
%----------------------------------------------------------
% CASO DE USO ABRIENDO CASINO
%----------------------------------------------------------
\label{lac}

	\begin{cu}{\ac}
	\descripxizq=1cm
		\defln{ Este caso de uso explica como se abre el casino }
		\actor{\admin}
		\pre{ El casino debe estar cerrado }
		\post{ El casino esta abierto}

	\begin{cursoe}{}
	
		\paso{1 El \admin ingresa al sistema la lista actualizada de los clientes }{}
		\paso{2 El \admin configura los valores de las fichas que se usaran en la jornada}{}
		\paso{3 El \admin configura el porcentaje de la jugada todos ponen }{}
		\paso{4 El \admin determina los valores minimos para la entrega de premios}{}
		\paso{5 El sistema informado que se ha iniciado un nuevo dia}{}
		
		\paso{ Fin C.U.}{}
		

	\end{cursoe}
\end{cu}

%----------------------------------------------------------
% CASO DE USO CERRANDO CASINO
%----------------------------------------------------------

\label{lccas}

\begin{cu}{\ccas}
	\descripxizq=1cm
		\defln{ Este caso de uso explica como se cierra el casino }
		\actor{\admin}
		\pre{ El casino debe estar abierto }
		\post{ El casino esta cerrado}

\begin{cursoe}{}
	
		\paso{1 El \admin cierra el casino }{}
		\pasoa{2 El sistema informa que el casino se cerrara}{2.1 El sistema notifica que hay mesas abiertas. No se puede 							cerrar. Ir a paso 3}
		\paso{ Fin C.U.}{}
		

	\end{cursoe}
\end{cu}

\newpage
%----------------------------------------------------------
%CASO DE USO ACTIVANDO MODO DIRIGIDO
%----------------------------------------------------------
\label{lactm}
\begin{cu}{\actm}
	\descripxizq=1cm
		\defln{Este caso de uso explica como un manipulador puede activar el modo dirigido del casino}
		\actor{\adm}
		\pre{ El casino debe estar abierto y desactivado el modo dirigido}
		\post{ Se ha activado el modo dirigido }

	\begin{cursoe}{}
	
		\paso{1 El manipulador selecciona activar el modo dirigido del casino }{}
		
		\paso{2 El sistema activa el modo dirigido, desactiva las jugadas azarosas}{}
						
		\paso{3 El sistema informa que se ha activado el modo dirigido del casino y solicita ingresar el resultado de las jugadas}{}
		
		\paso{4 El manipulador ingresa el resultado de las jugadas}{}
		
  	\paso{5 El sistema modifica la configuraci�n del casino}{}	
		
		\paso{6 El sistema informa que el casino se encuentra ahora en modo dirigido }{}

		\paso{7 Fin C.U.}{}
		
	\end{cursoe}
\paragraph{Comentario: }
\begin{flushleft}
Nota al Dpto de Sistemas: asumimos que el ingreso de los resultados de las jugadas es una configuraci�n dada
%me refiero a q ingresa un "archivo" con cierta configuracion ya armada para los result
\end{flushleft}
\end{cu}

%----------------------------------------------------------
%Caso de uso Desactivando modo dirigido
%----------------------------------------------------------
\label{ldesm}
\begin{cu}{\desm}
	\descripxizq=1cm
		\defln{Este caso de uso explica como se desactiva el modo dirigido }
	  \actor{\adm}
		\pre{ El casino debe estar abierto y activado el modo dirigido }
		\post{ Se ha desactivado el modo dirigido y el resultado de las jugadas ser� aleatorio }
		
	\begin{cursoe}{}
	
		\paso{1 El manipulador selecciona desactivar el modo dirigido del casino}{}
		
		\paso{2 El sistema desactiva el modo dirigido activando los resultados azarosos para todas las jugadas y asi tambien las ocurrencias de los distintos tipos de jugadas }{}
		
		\paso{3 El sistema informa que el casino se encuentra ahora en modo normal }{}
		
		\paso{4 Fin C.U.}{}

	\end{cursoe}
	
\end{cu}


\newpage
%----------------------------------------------------------
%Caso de uso JUGADA FELIZ
%----------------------------------------------------------
\label{ljugf}
\begin{cu}{\jugf}
	\descripxizq=1cm
		\defln{ Este caso de uso explica como se seleccionara la aparici�n de las jugadas felices }
		\actor{\adm}
		\pre{ El casino est� abierto y esta activado el modo dirigido }
		\post{ Se ha generado una nueva aparicion de jugada feliz }

	\begin{cursoe}{}
	
		\paso{1 El \adm selecciona un juego }{}
		
		\paso{2 El \adm selecciona una jugada particular }{}
		
		\paso{3 El sistema genera una nueva jugada feliz }{3.1 ERROR El sistema informa que la jugada se solapa con otra jugada feliz en el mismo instante en el casino o con otra jugada todos ponen en el mismo juego y jugada}
	
		\paso{4 Fin C.U.}{}
		

	\end{cursoe}

\end{cu}

%----------------------------------------------------------
%Caso de uso TODOS PONEN
%----------------------------------------------------------
\label{ljugtp}
\begin{cu}{\jugtp}
	\descripxizq=1cm
		\defln{ Este caso de uso explica como se seleccionara la aparici�n de las jugadas todos ponen }
		\actor{\adm}
		\pre{ El casino est� abierto y esta activado el modo dirigido }
		\post{ Se ha generado una nueva aparicion de jugada todos ponen }

	\begin{cursoe}{}
	
		\paso{1 El \adm selecciona un juego }{}
		
		\paso{2 El \adm selecciona las jugadas que se veran afectadas por la jugada todos ponen}{}
		
		\paso{3 El sistema genera una nueva jugada todos ponen }{}

		\paso{4 Fin C.U.}{}
		
	\end{cursoe}
\end{cu}	

\newpage
%----------------------------------------------------------
% CASO DE USO JUGANDO TRAGAMONEDAS
%----------------------------------------------------------
\label{ljtra}
\begin{cu}{\jtra}
	\descripxizq=1cm
		\defln{Este caso de uso explica como un \jutra que ha ingresado en el casino puede jugar en una mesa de tragamonedas}
		\actor{\jutra}
		\pre{El \jutra debe estar dentro del casino y no estar jugando en ninguna otra mesa}
		\post{El \jutra ha jugado al tragamonedas y ha salido del juego, si ha resultado ganador se modifica su saldo}

	\begin{cursoe}{}
	
		\paso{1 El \jutra selecciona el valor de la ficha de la nueva mesa de tragamonedas}{}
		
		\paso{2 El sistema crea una nueva mesa para el \jutra}{}
			
		\paso{3 El sistema informa que el jugador ha ingresado a una mesa de tragamonedas.}{}
		
		\paso{4 El jugador apuesta \incl {\apotra}}{}
				
		\paso{5 El \jutra decide comenzar la jugada }{5.1 Error, la apuesta no es valida, ir a 11}
		
		\paso{6 El sistema da comienzo al juego }{}	
		
		\paso{7 El sistema informa si la jugada es normal, feliz o todos ponen}{}	
	  
	  \paso{8 El sistema muestra el resultado del juego}{}
	  
	  \paso{9 Si el resultado es ganador el sistema acredita el premio correspondiente }{}
	  
	  \paso{10 Si el jugador desea seguir jugando ir a paso 4}{}
	  
	  \paso{11 Sino el sistema automaticamente cierra la mesa}{}
	  
		\paso{12 Fin C.U.}{}
		
  \end{cursoe}

\end{cu}

\newpage
%-------------------------------
%CASO DE USO APOSTANDO EN TRAGAMONEDAS
%--------------------------------
\label{lapotra}
\begin{cu}{\apotra}
	\descripxizq=1cm
		\defln{Este caso de uso explica como un \aptra apuesta en una mesa Tragamonedas }
		\actor{\apotra}
		\pre{El \jutra debe haber ingresado a una mesa Tragamonedas}
		\post{El \apotra ha apostado en el tragamonedas, y su saldo ha sido modificado }

	\begin{cursoe}{}
    \paso{1 El \apotra selecciona la cantidad de fichas a apostar (entre las permitidas) }{}
		
		\paso{2 Si el \apotra no es VIP el sistema verifica si tiene saldo suficiente para la apuesta}{}
			
		\paso{3 El sistema valida la apuesta y descuenta el monto del saldo del jugador}{3.1 Error: El sistema informa que no tiene saldo suficiente para realizar esta apuesta. Ir a Fin C.U.}
				
		\paso{4 El sistema incrementa el pozo "Premio Gordo Progresivo" }{}
		
		\paso{5 Fin C.U.}{}	
		
	\end{cursoe}
\end{cu}
		
%----------------------------------------------------------
% CASO DE USO Ingresando a la mesa de craps
%----------------------------------------------------------
\label{lincr}
\begin{cu}{\incr}
	\descripxizq=1cm
		\defln{Este caso de uso explica como un jugador puede unirse o abrir una nueva mesa de craps }
		\actor{\jucr}
		\pre{El \jucr debe estar dentro del casino y no estar jugando en ninguna otra mesa}
		\post{El \jucr ha ingresado en una mesa de Craps }

	\begin{cursoe}{}
		
		\paso{1 Si hay mesas disponibles el sistema pregunta si el \jucr desea unirse a una mesa, sino hay mesas disponibles ir a paso 3}{}
		
		\paso{2 Si el jugador decide unirse a una mesa selecciona la mesa a la cual desea unirse. Ir a 4}{}
		
		\paso{3 El sistema crea una nueva mesa para el \jucr }{}
		
		\paso{4 El sistema informa que el jugador ha ingresado a una mesa de Craps.}{}
		
	  \paso{5 Fin C.U.}{}
		
	\end{cursoe}
	
\end{cu}

\newpage
%----------------------------------------------------------
% CASO DE USO Jugando Craps
%----------------------------------------------------------
\label{ljcr}
\begin{cu}{\jcr}
	\descripxizq=1cm
		\defln{Este caso de uso explica como un jugador juega en una nueva mesa de craps }
		\actor{\jucr}
		\pre{El \jucr debe estar dentro del casino y haber ingresado en una mesa de Craps}
		\post{El \jucr ha jugado Craps }

	\begin{cursoe}{}
		
		\paso{1 Si el \jucr desea apuesta. \exti{\apcr}}{}
		
		\paso{2 Si es su turno tira los dados sino observa el resultado}{2.1 Error: La apuesta no es valida por falta de saldo. Ir a 4}%lo mande a ese paso xq si era el unico jugador
		%y no tiene saldo el sistema va a mostrar un resultado "nulo" hasta q se aburra
		
		\paso{3 El sistema informa si la jugada es normal, feliz o todos ponen}{}
				
		\paso{4 El sistema muestra los resultados}{}
		
	  \paso{5 Si el \jucr ha ganado alguna apuesta el sistema acredita el premio correspondiente}{}
	  
	  \paso{6 El sistema modifica el saldo del \jucr si corresponde}{}
	  
	  \paso{7 Si no desea seguir jugando el sistema informa que perdera todas las apuestas  que tenga activas.}{}
% Chicos: lo puse en este orden xq capaz despues de avisarle q pierde sus apuestas si se retira, se arrepiente y en el paso 8 puede optar x seguir, pero si no les gusta lo pueden varia	  
	  \paso{8 Si el jugador desea seguir jugando ir a 1 }{}
	  
	  \paso{9 Sino el sistema saca de la mesa de Craps al \jucr. Incluye saliendo craps }{}
	  
	  \paso{10 Fin C.U.}{}
		
	\end{cursoe}
	
\end{cu}

%----------------------------------------------------------
% CASO DE USO APOSTANDO CRAPS
%----------------------------------------------------------
\textcolor{red}{ CU apostando}

%----------------------------------------------------------
% CASO DE USO Jugando Craps
%----------------------------------------------------------
\label{lscr}
\begin{cu}{\scr}
	\descripxizq=1cm
		\defln{Este caso de uso explica como un jugador sale de una mesa de craps }
		\actor{\jucr}
		\pre{El \jucr debe estar dentro del casino y haber ingresado en una mesa de Craps}
		\post{El \jucr ha salido de una mesa craps }

	\begin{cursoe}{}
		\paso{1 El jugador desea no seguir jugando  }{}
	  
	  \paso{2 El sistema saca de la mesa de Craps al \jucr }{}
	  
	  \paso{3 El sistema cierra automaticamente la mesa si este era el unico jugador en ella}{}	  
	  
	  \paso{4 Fin C.U.}{}

	\end{cursoe}
\end{cu}

	
\textcolor{red}{ CU ver aca para abajo los CU}
	
%--------------------------------------------------------------------------------	

\begin{cu}{\infomovjug}
	\descripxizq=1cm
		\defln{ Este caso de uso explica la generacion del reporte Movimientos por Jugador }
		\actor{\empmark}
		\pre{ El casino debe estar abierto }
		\post{ El sistema generara el reporte Movimientos por Jugador}

\begin{cursoe}{}
	
		\paso{1 El \empmark solicita el reporte "Movimientos Por Jugador" }{}
		\paso{2 El sistema devuelve tal reporte}{}
		\paso{ Fin C.U.}{}
		

	\end{cursoe}
\end{cu}

\begin{cu}{\infoestact}
	\descripxizq=1cm
		\defln{ Este caso de uso explica la generacion del reporte Estado Actual }
		\actor{\empmark}
		\pre{ El casino debe estar abierto }
		\post{ El sistema generara el reporte Estado Actual}

\begin{cursoe}{}
	
		\paso{1 El \empmark solicita el reporte "Estado Actual" }{}
		\paso{2 El sistema devuelve tal reporte}{}
		\paso{ Fin C.U.}{}
		

	\end{cursoe}
\end{cu}

\begin{cu}{\inforankjug}
	\descripxizq=1cm
		\defln{ Este caso de uso explica la generacion del reporte Ranking de Jugadores }
		\actor{\empmark}
		\pre{ El casino debe estar abierto }
		\post{ El sistema generara el reporte Ranking de Jugadores}

\begin{cursoe}{}
	
		\paso{1 El \empmark solicita el reporte "Ranking de Jugadores" }{}
		\paso{2 El sistema devuelve tal reporte}{}
		\paso{ Fin C.U.}{}
		

	\end{cursoe}
\end{cu}


%
\begin{cu}{\infdmj}
	\descripxizq=1cm
		\defln{ Este caso de uso explica como el emk solicita informe de detalles de movimientos por jugador }
		\actor{\adm}
		\pre{  }
		\post{ El sistema ha generado el informe }

	\begin{cursoe}{}
	
		\paso{1 El \emk solicita informe "Detalles de movimientos por jugador" }{}
		
		\paso{2 El sistema genera el informe de "Detalles de movimientos por jugador" }{}
		
		\paso{3 Fin C.U.}{}
		

	\end{cursoe}
%\paragraph{Comentario: }
%\begin{flushleft}
%\end{flushleft}
\end{cu}