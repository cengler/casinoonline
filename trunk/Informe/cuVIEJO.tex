%\newcommand{\JugTra}{{\bf Jugando Tragamonedas }}
%\newcommand{\JugCra}{{\bf Jugando Craps }}
%\newcommand{\apo}{{\bf Apostando }}
%\newcommand{\j}{{\bf Jugador }}
%

%Asi se escriben los "extiende"
%\exti {Jugando tragamonedas}

%Actores
\newcommand{\pc}{{\bf Cliente del casino }}
\newcommand{\adm}{{\bf Administrador }}
\newcommand{\ptra}{{\bf Potencial jugador de tragamonedas }}
\newcommand{\jutra}{{\bf Jugador de Tragamonedas }}

%----------------------
%CU
\newcommand{\ic}{{\bf Ingresando a casino }}
\newcommand{\salc}{{\bf Saliendo del casino }}
\newcommand{\atra}{{\bf Abriendo Mesa Tragamonedas }}
\newcommand{\jtra}{{\bf Jugando Tragamonedas }}
\newcommand{\actm}{{\bf Activando Modo Dirigido }}
\newcommand{\desm}{{\bf Desactivando Modo Dirigido }}
\newcommand{\jugf}{{\bf Jugada Feliz }}
\newcommand{\jugtp}{{\bf Jugada Todos Ponen }}
\newcommand{\ac}{{\bf Abriendo Casino }}
\newcommand{\cc}{{\bf Cerando Casino }}
%\newcommand{\}{{\bf  }}












%-----------------------------------------------------------
%----------------------------------------------------------------------
% O ESTO???????
%\begin{cu}{\cuic}
%	\descripxizq=1cm
%		\defln{Este caso de uso explica como un \cl interactua con el sistama para entrar al casino }
%		\actor{\cl}
%		\pre{true}
%		\post{El \cl ingres� al casino online}
%
%	\begin{cursoe}{}
%	
%		\paso{1 El \cl ingresa DNI y password }{}
%			
%		\paso{2 El sistema verifica los datos ingresados corresponden a un cliente del casino}{}
%		
%		\paso{3 El cliente ingres� al casino}{3.1 Si el DNI no corresponde a ningun cliente registrado, ir a paso 4}
%		
%		\pasoa{3.2 Si el password no corresponde al DNI ingresado ir a paso 4}
%		
%		\paso{4 Fin de caso de uso}{}
%		
%	\end{cursoe}
%
%		\paragraph{Preguntas:}
%		\begin{flushleft}
%		Lista puede contener DNI y password?
%		\end{flushleft}
%	
%\end{cu}
%--------------------------


%-------------------------------------------------------------------
%Aca esta el viejo cu abrir mesa o lo q quedo de el por si lo ncesitabamos	
%\begin{cursoe}{}
%	
%		\paso{1 El \ptra selecciona el valor de la ficha de la nueva mesa de tragamonedas}{}
%		
%		\paso{2 El sistema crea una nueva mesa para ese jugador}{}
%			
%		\paso{3 El sistema informa que el jugador ha ingresado a una mesa de tragamonedas}{}
%		
%		\paso{4  }{}
%		
%		\paso{5 Fin C.U.}{}
%		
%	\end{cursoe}
%

%-----------
%----------------------------------------------------------------------
%aCA FALTARIA AGREGAR LO Q HIZO CHRIS DE CRAPS
%---------

\begin{cu}{\actm}
	\descripxizq=1cm
		\defln{Este caso de uso explica como un administrador puede activar el modo dirigido del casino}
		\actor{\adm}
		\pre{ El casino debe estar abierto y desactivado el modo dirigido}
		\post{ Se ha activado el modo dirigido }

	\begin{cursoe}{}
	
		\paso{1 El administrador selecciona activar el modo dirigido del casino }{}
		
		\paso{2 El sistema activa el modo dirigido, desactiva las jugadas azarosas}{}
						
		\paso{3 El sistema informa que se ha activado el modo dirigido del casino y solicita ingresar el resultado de las jugadas}{}
		
		\paso{4 El administrador ingresa el resultado de las jugadas???}{}
% eso suena horrible!!!! q hacemos???		
		\paso{5 Fin C.U.}{}
		
	\end{cursoe}
%\paragraph{Comentario: }
%\begin{flushleft}
%\end{flushleft}
\end{cu}
%-----------------------------

\begin{cu}{\desm}
	\descripxizq=1cm
		\defln{Este caso de uso explica como se desactiva el modo dirigido }
	  \actor{\adm}
		\pre{ El casino debe estar abierto y activado el modo dirigido }
		\post{ Se ha desactivado el modo dirigido y el resultado de las jugadas ser� aleatorio }
		
	\begin{cursoe}{}
	
		\paso{1 El administrador selecciona desactivar el modo dirigido del casino}{}
		
		\paso{2 El sistema desactiva el modo dirigido activando los resultados azarosos para todas las jugadas y asi tambien las ocurrencias de los distintos tipos de jugadas }{}
		
		\paso{3 El sistema informa que el casino se encuentra ahora en modo normal }{}
		
		\paso{4 Fin C.U.}{}
		

	\end{cursoe}


%\paragraph{Comentario: }
%\begin{flushleft}
%\end{flushleft}
\end{cu}
%----------------------------------------
\begin{cu}{\jugf}
	\descripxizq=1cm
		\defln{ Este caso de uso explica como se seleccionara la aparici�n de las jugadas felices }
		\actor{\adm}
		\pre{ El casino est� abierto y esta activado el modo dirigido }
		\post{ Se ha generado una nueva aparicion de jugada feliz }

	\begin{cursoe}{}
	
		\paso{1 El \adm selecciona un juego }{}
		
		\paso{2 El \adm selecciona una jugada particular }{}
		
		\paso{3 El sistema genera una nueva jugada feliz }{3.1 ERROR El sistema informa que la jugada se solapa con otra jugada feliz en el casino o jugada todos ponen en el mismo juego y jugada}
%no estoy inspirada!!!!		
		\paso{ Fin C.U.}{}
		

	\end{cursoe}


%\paragraph{Comentario: }
%\begin{flushleft}
%\end{flushleft}
\end{cu}

%---------------------------------

\begin{cu}{\jugtp}
	\descripxizq=1cm
		\defln{ Este caso de uso explica como se seleccionara la aparici�n de las jugadas felices }
		\actor{\adm}
		\pre{ El casino est� abierto y esta activado el modo dirigido }
		\post{ Se ha generado una nueva aparicion de jugada todos ponen }

	\begin{cursoe}{}
	
		\paso{1 El \adm selecciona un juego }{}
		
		\paso{2 El \adm selecciona una jugada particular }{}
		
		\paso{3 El sistema genera una nueva jugada feliz }{3.1 ERROR El sistema informa que la jugada se solapa con una jugada feliz o jugada todos ponen en el mismo juego y jugada}

		\paso{4 Fin C.U.}{}
		

	\end{cursoe}
%\paragraph{Comentario: }
%\begin{flushleft}
%\end{flushleft}
\end{cu}
%-------------------------------------------------------------------------
\begin{cu}{\ac}
	\descripxizq=1cm
		\defln{ Este caso de uso explica como se abre el casino }
		\actor{\adm}
		\pre{ El casino debe estar cerrado }
		\post{ El casino esta abierto}

	\begin{cursoe}{}
	
		\paso{1 El \adm }{}
		
		
		\paso{ Fin C.U.}{}
		

	\end{cursoe}


%\paragraph{Comentario: }
%\begin{flushleft}
%\end{flushleft}
\end{cu}

%---------------------------------
%Copiar y pegar este luego completar!!!!!!!!
\begin{cu}{\}
	\descripxizq=1cm
		\defln{}
		\actor{}
		\pre{ }
		\post{ }

	\begin{cursoe}{}
	
		\paso{1 }{}
		
		
		\paso{ Fin C.U.}{}
		

	\end{cursoe}


%\paragraph{Comentario: }
%\begin{flushleft}
%\end{flushleft}
\end{cu}

%---------------------------------
%aca esta el viejo cu jugando q lo incorpore con apostando

%\begin{cu}{\jtra}
%	\descripxizq=1cm
%		\defln{Este caso de uso explica como un cliente que ha ingresado en el casino y ha abierto una mesa de tragamonedas juega y apuesta}
%		\actor{\jutra}
%		\pre{El \jutra debe haber ingresado al casino y abierto una mesa de tragamonedas}
%		\post{El \jutra ha jugado en la mesa tragamonedas y su saldo se ha modificado }
%
%	\begin{cursoe}{}
%	
%		\paso{1 El \jutra selecciona la cantidad de fichas a apostar}{}
%		
%		\paso{2 Si el jugador no es VIP el sistema verifica si tiene saldo suficiente para la apuesta}{}
%			
%		\paso{3 Si el saldo es sufiente el sistema valida la apuesta, sino el sistema informa que no tiene saldo suficiente para realizar esta apuesta. Ir a Fin C.U.}{}
%				
%		\paso{4 El sistema incrementa el pozo "Premio Gordo Progresivo" }{}
%		
%		\paso{4 El sistema informa si la jugada es normal, feliz o todos ponen}{}
%		
%	  \paso{5 El sistema da comienzo al juego }{}		
%	  
%	  \paso{6 El sistema muestra el resultado del juego }{}
%	  
%	  \paso{7 Si el resultado es ganador el sistema acredita el premio correspondiente }{}
%
%		\paso{ Fin C.U.}{}
%		
%		\paso{5 Fin C.U.}{}
%		
%		
%		
%	\end{cursoe}
%
%
%%\paragraph{Preguntas: }
%%\begin{flushleft}
%
%%\end{flushleft}
%
%\end{cu}
%
%ESTE NO SIRVE!!!!!!!!!
%
%\begin{cu}{\cucc}
%	\descripxizq=1cm
%		\defln{Este caso de uso explica como un cliente loggeado interact�a con el sistema con el fin de comprar credito que le permitir� apostar en el casino}
%		\actor{\cc}
%		\pre{el \cc esta loggeado en el sistema }
%		\post{Al \cc se le asinga la cantidad de credito comprado, su saldo real se disminnuye correpondientemente y ese saldo se le asigna al saldo real del casino }
%
%	\begin{cursoe}{}
%	
%		\paso{1 El \cc ingresa el monto que desea invertir para hacer sus apuestas.}{}
%			
%		\paso{2 El sistema verifica si el \cc posee saldo suficiente.}{}
%		
%		\paso{3 El saldo real del \cc disminuye en la cantidad invertida.}{3.1 El \cc no posee saldo real suficiente para realizar la compra de cr�dito. Ir a paso 7}
%		
%		\paso{4 Es saldo real del casino aumenta en la cantidad invertida por el \cc}{}
%		
%		\paso{5 El sisteme le asigna el credito correspondiente a la compra}{}
%		
%		\paso{6 El sistema notifica al \cc que la operaci�n de compra se realiz� exitosamente.}{}
%
%		\paso{7 Fin de caso de uso.}{}
%		
%	\end{cursoe}
%
%\paragraph{Preguntas: }
%\begin{flushleft}
%�Hay que hablar del saldo del casino?
%\end{flushleft}
%
%\end{cu}
%
%
%
%
%
%\newpage
%
%
%
%
%
%
%
%\begin{cu}{\scas}
%	\descripxizq=1cm
%		\defln{Este caso de uso explica la interaccion del \salcas con el sistema, cuando decidi� retirarse del casino}
%		\actor{\salcas}
%		\pre{el \salcas esta loggeado en el sistema }
%		\post{El \salcas se retiro del casino y su saldo real se incremento en la cantidad correspondiente al credito que poseia }
%
%	\begin{cursoe}{}
%	
%		\paso{1 Si el \salcas se encuentra jugando algun juego, se le da la opcion de abandonar y perder todo el credito apostado en dicho juego}{}
%		
%		\paso{2 El sistema transfiere del saldo del casino al saldo del usuario la cantidad de dinero correspondiente al credito que posee el \salcas }{2.1 El \salcas decidi� permanecer en el juego. Ir a paso 4}
%		\paso{3 Se le notificas al \salcas que ha salido exitosamente del sistema}{}
%
%		\paso{4 Fin de caso de uso.}{}
%		
%	\end{cursoe}
%	
%	
%\paragraph{Preguntas: }
%\begin{flushleft}
%�Se puede poner condiciones en el curso de ej?\\
%�Se puede poner un ir a en el curso normal?\\
%�Hay que hablar del saldo del casino?
%\end{flushleft}
%\end{cu}
%
%\newpage
%
%
%\begin{cu}{\habmes}
%	\descripxizq=1cm
%		\defln{Este caso de uso explica como un cliente logeado habilita una mesa}
%		\actor{\hm}
%		\pre{el \hm esta loggeado en el sistema}
%		\post{el \hm habilita una mesa y se encuenta anotado la misma }
%
%	\begin{cursoe}{}
%	
%		\paso{1}{}
%			
%		\paso{2}{}
%		
%		\paso{3}{}
%		
%		\paso{4}{}
%		\paso{5}{}
%		
%	\end{cursoe}
%\end{cu}
%
%\newpage
%
%
%
%
%
%\newpage
%
%\paragraph{Glosario}
%
%\begin{itemize}
%	\item \textbf{Casino: } Sistema online del casino
%	\item \textbf{Casino Online: } Sistema online del casino
%	\item \textbf{Saldo real: } Dinero en efectivo que posee en cliente
%\end{itemize}