\documentclass[a4paper, 10pt, notitlepage]{article}

\usepackage{moreverb} %para importar codigo

\usepackage{pepotina} %paquete personal para la caratula del DC

\usepackage[spanish,activeacute]{babel}
\usepackage{babel} %paquete de idioma

\usepackage[latin1]{inputenc}

\usepackage{color}

\usepackage{caeycaeING}

\usepackage{fancyhdr} %linea sup con comentarios

%\usepackage{listings}

\usepackage{lscape} %para hoja apaisada

\usepackage{framed} %para crear cajas de texto

\usepackage{lastpage} %ultima pagina

\usepackage{pstricks}
\usepackage{uml} %UML


\addtolength{\topmargin}{-50pt} 
\addtolength{\textwidth}{105pt}
\addtolength{\textheight}{120pt}
\addtolength{\oddsidemargin}{-50pt}

%\newcommand{\minix}{\textsl{minix }}

%%% Encabezado y pie de p'agina
\pagestyle{fancy}
\fancyhead[LO]{Ingenieria del Software I}
\fancyhead[C]{}
\fancyhead[RO]{P\'agina \thepage\ de \pageref{LastPage}}
\renewcommand{\headrulewidth}{0.4pt}
\fancyfoot{}

\newcommand{\falta}{ \begin{framed}	\begin{center} \hspace{1cm} \Large FALTA \normalsize \hspace{1cm} \end{center} \end{framed}}

\begin{document}

\universidad{Universidad de Buenos Aires}
\facultad{Facultad de ciencias exactas y naturales}
\departamento{Departamento de Computacion}
\materia{Ingenieria del Software I}
\resumen{En este trabajo...}
\keys{UML, Objetivos}
\titulo{Trabajo Practico}
\subtitulo{Grupo N� 2}
\fecha{1er Cuatrimeste 2008}
\footspace{1cm}
\integrante{Aquino, Isis}{caeycae@gmail.com}
\integrante{Alvarez, Maria de los Angeles}{mdelosaalvarez@gmail.com}%264/05
\integrante{Garcia, Ana Daniela}{caeycae@gmail.com}
\integrante{Engler, Christian Alejandro}{caeycae@gmail.com}%314/05

%caratula
%\maketitle{} %OK

%\tableofcontents
%\newpage


\newcommand{\cuic}{{\bf Ingresando a casino }}
\newcommand{\cl}{{\bf Cliente }}
\newcommand{\cucc}{{\bf Comprando credito }}
\newcommand{\cc}{{\bf Comprador de credito }}
\newcommand{\habmes}{{\bf Habilitando mesa}}
\newcommand{\hm}{{\bf Habilitador de mesa }}
\newcommand{\scas}{{\bf Retirandose del Casino }}
\newcommand{\salcas}{{\bf Saliente del Casino }}
\newcommand{\pc}{{\bf posible cliente }}







\begin{cu}{\cucc}
	\descripxizq=1cm
		\defln{Este caso de uso explica como un cliente loggeado interact�a con el sistema con el fin de comprar credito que le permitir� apostar en el casino}
		\actor{\cc}
		\pre{el \cc esta loggeado en el sistema }
		\post{Al \cc se le asinga la cantidad de credito comprado, su saldo real se disminnuye correpondientemente y ese saldo se le asigna al saldo real del casino }

	\begin{cursoe}{}
	
		\paso{1 El \cc ingresa el monto que desea invertir para hacer sus apuestas.}{}
			
		\paso{2 El sistema verifica si el \cc posee saldo suficiente.}{}
		
		\paso{3 El saldo real del \cc disminuye en la cantidad invertida.}{3.1 El \cc no posee saldo real suficiente para realizar la compra de cr�dito. Ir a paso 7}
		
		\paso{4 Es saldo real del casino aumenta en la cantidad invertida por el \cc}{}
		
		\paso{5 El sisteme le asigna el credito correspondiente a la compra}{}
		
		\paso{6 El sistema notifica al \cc que la operaci�n de compra se realiz� exitosamente.}{}

		\paso{7 Fin de caso de uso.}{}
		
	\end{cursoe}

\paragraph{Preguntas: }
\begin{flushleft}
�Hay que hablar del saldo del casino?
\end{flushleft}

\end{cu}





\newpage







\begin{cu}{\scas}
	\descripxizq=1cm
		\defln{Este caso de uso explica la interaccion del \salcas con el sistema, cuando decidi� retirarse del casino}
		\actor{\salcas}
		\pre{el \salcas esta loggeado en el sistema }
		\post{El \salcas se retiro del casino y su saldo real se incremento en la cantidad correspondiente al credito que poseia }

	\begin{cursoe}{}
	
		\paso{1 Si el \salcas se encuentra jugando algun juego, se le da la opcion de abandonar y perder todo el credito apostado en dicho juego}{}
		
		\paso{2 El sistema transfiere del saldo del casino al saldo del usuario la cantidad de dinero correspondiente al credito que posee el \salcas }{2.1 El \salcas decidi� permanecer en el juego. Ir a paso 4}
		\paso{3 Se le notificas al \salcas que ha salido exitosamente del sistema}{}

		\paso{4 Fin de caso de uso.}{}
		
	\end{cursoe}
	
	
\paragraph{Preguntas: }
\begin{flushleft}
�Se puede poner condiciones en el curso de ej?\\
�Se puede poner un ir a en el curso normal?\\
�Hay que hablar del saldo del casino?
\end{flushleft}
\end{cu}

\newpage


\begin{cu}{\habmes}
	\descripxizq=1cm
		\defln{Este caso de uso explica como un cliente logeado habilita una mesa}
		\actor{\hm}
		\pre{el \hm esta loggeado en el sistema}
		\post{el \hm habilita una mesa y se encuenta anotado la misma }

	\begin{cursoe}{}
	
		\paso{1}{}
			
		\paso{2}{}
		
		\paso{3}{}
		
		\paso{4}{}
		\paso{5}{}
		
	\end{cursoe}
\end{cu}

\newpage



\begin{cu}{\cuic}
	\descripxizq=1cm
		\defln{Este caso de uso explica como un \cl interactua con el sistama para entrar al casino }
		\actor{\cl}
		\pre{true}
		\post{El \cl ingres� al casino online}

	\begin{cursoe}{}
	
		\paso{1 El \cl ingresa DNI y password }{}
			
		\paso{2 El sistema verifica los datos ingresados corresponden a un cliente del casino}{}
		
		\paso{3 El cliente ingres� al casino}{3.1 Si el DNI no corresponde a ningun cliente registrado, ir a paso 4}
		
		\pasoa{3.2 Si el password no corresponde al DNI ingresado ir a paso 4}
		
		\paso{4 Fin de caso de uso}{}
		
	\end{cursoe}

		\paragraph{Preguntas:}
		\begin{flushleft}
		Lista puede contener DNI y password?
		\end{flushleft}
	
\end{cu}


\newpage

\paragraph{Glosario}

\begin{itemize}
	\item \textbf{Casino: } Sistema online del casino
	\item \textbf{Casino Online: } Sistema online del casino
	\item \textbf{Saldo real: } Dinero en efectivo que posee en cliente
\end{itemize}

\end{document}