\documentclass[a4paper, 10pt, notitlepage]{article}

\usepackage{moreverb} %para importar codigo

\usepackage{pepotina} %paquete personal para la caratula del DC

\usepackage[spanish,activeacute]{babel}
\usepackage{babel} %paquete de idioma

\usepackage[latin1]{inputenc}

\usepackage{color}

\usepackage{caeycaeING}

\usepackage{fancyhdr} %linea sup con comentarios

%\usepackage{listings}

\usepackage{lscape} %para hoja apaisada

\usepackage{framed} %para crear cajas de texto

\usepackage{lastpage} %ultima pagina

\usepackage{pstricks}
\usepackage{uml} %UML


\addtolength{\topmargin}{-50pt} 
\addtolength{\textwidth}{105pt}
\addtolength{\textheight}{120pt}
\addtolength{\oddsidemargin}{-50pt}

%\newcommand{\minix}{\textsl{minix }}

%%% Encabezado y pie de p'agina
\pagestyle{fancy}
\fancyhead[LO]{Ingenieria del Software I}
\fancyhead[C]{}
\fancyhead[RO]{P\'agina \thepage\ de \pageref{LastPage}}
\renewcommand{\headrulewidth}{0.4pt}
\fancyfoot{}

\newcommand{\falta}{ \begin{framed}	\begin{center} \hspace{1cm} \Large FALTA \normalsize \hspace{1cm} \end{center} \end{framed}}

\begin{document}

\universidad{Universidad de Buenos Aires}
\facultad{Facultad de ciencias exactas y naturales}
\departamento{Departamento de Computacion}
\materia{Ingenieria del Software I}
\resumen{En este trabajo...}
\keys{UML, Objetivos}
\titulo{Trabajo Practico}
\subtitulo{Grupo N� 2}
\fecha{1er Cuatrimeste 2008}
\footspace{1cm}
\integrante{Aquino, Isis}{caeycae@gmail.com}
\integrante{Alvarez, Maria de los Angeles}{mdelosaalvarez@gmail.com}%264/05
\integrante{Garcia, Ana Daniela}{caeycae@gmail.com}
\integrante{Engler, Christian Alejandro}{caeycae@gmail.com}%314/05

%caratula
%\maketitle{} %OK

%\tableofcontents
%\newpage

\newcommand{\pc}{{\bf posible cliente }}

\begin{cu}{Dando de alta cliente}
	\descripxizq=1cm
		\defln{Este caso de uso explica como el \pc interactua con el sistema con el fin de crear una cuenta de usuario que le permita acceder en el casino}
		\actor{\pc}
		%\actorsec{}
		\pre{true}
		\post{el \pc dispone de una cuenta de usuario que le permita acceder en el casino}
	
	\begin{cursoe}{}
	
		\paso{1 El \pc ingresa al (sistema) pagina web del casino}
		{}
		
		\paso{2 El \pc ingresa su nombre, numero de documento, nick y password}
		{}
		
		\paso{3 \pc ingresa el numero de su cuenta corriente}
		{3.1 El sistema rechaza al \pc por ser menor de edad. Ir a paso 7}
		
		\pasoa{3.2 El sistema rechaza al \pc porque hay un usuario con el mismo nick. Ir a paso 2}
		
		\pasoa{3.3 El sistema rechaza la creacion del \pc porque hay un usuario con el mismo numero de documento. Ir a paso 7}
		
		\pasoa{3.4 El sistema rechaza la creacion del \pc porque el password no cumple con las normas de seguridad requeridas. Ir a paso 2}
		
		\paso{4 El sistema verificar� la existencia de la cuenta corriente ingresada. \incl{Verificando cuenta} }
		{}
		
		\paso{5 El sistema almacena la informaci�n del nuevo \pc.}
		{5.1 En caso de que la verificacion sea erronea ir a paso 7}
		
		\paso{6 Si el \pc lo deseea, ingresa al casino online \exti{Ingresando a casino}}
		{}
		
		\paso{7 Fin de caso de uso}{}
		
	\end{cursoe}
	
	\paragraph{Preguntas:}
		\begin{flushleft}
		�Debe ser mayor de ciarta edad para jugar?\\
		\end{flushleft}
	
\end{cu}


\newpage


\newcommand{\cucf}{{\bf Comprando fichas }}
\newcommand{\cf}{{\bf Comprador de fichas }}
\newcommand{\scc}{{\bf Sistema de cuentas corrientes }}
\newcommand{\caja}{Caja de fichas}

\begin{cu}{\cucf}
	\descripxizq=3cm
		\defln{Este caso de uso explica como un cliente loggeado interact�a con el sistema con el fin de comprar fichas que le permitir�n jugar en el casino}
		\actor{\cf}
		\actorsec{\scc}
		\pre{el \cf esta loggeado en el sistema }
		\post{las fichas compradas por el \cf son agregadas a su \caja}

	\begin{cursoe}{}
	
		\paso{1 El \cf selecciona cu�ntas fichas de cada tipo desea comprar.}{}
			
		\paso{2 El sistema establece conexi�n con el \scc.}{}
		
		\paso{3 El sistema transfiere el monto total de la compra desde la cuenta corriente del \cf a la cuenta del casino.}
		{3.1 No se puede establecer la conexi�n con el \scc. Ir a paso 6.}
		
		\paso{4 El sistema agrega a la \caja del \cf las fichas compradas.}
		{4.1 Los fondos de la cuenta corriente del \cf son insuficientes para realizar la operaci�n. Ir a paso 6}
				
		\paso{5 El sistema notifica al \cf que la operaci�n de compra se realiz� exitosamente.}{}

		\paso{6 Fin de caso de uso.}{}
		
	\end{cursoe}
%	
%	\paragraph{Preguntas:}
%		\begin{flushleft}
%		�Debe ser mayor de cierta edad para jugar?
%		\end{flushleft}
%	
\end{cu}


\newpage

\newcommand{\cujj}{{\bf Jugando juegos }}
\newcommand{\jj}{{\bf Jugandor de juegos }}

\newcommand{\cuic}{{\bf Ingresando a casino }}
\newcommand{\cl}{{\bf Cliente }}

\begin{cu}{\cuic}
	\descripxizq=1cm
		\defln{Este caso de uso explica como un \cl interactua con el sistama para loggerase en el casino }
		\actor{\cl}
		\pre{true }
		\post{El \cl ingres� al casino online}

	\begin{cursoe}{}
	
		\paso{1 El \cl ingresa nick y password }{}
			
		\paso{2 El sistema verifica los datos ingresados}{}
		
		\paso{3 El cliente ingres� al casino}{3.1 Si el nick no existe ir a paso 1}
		
		\pasoa{3.2 Si el password no corresponde al nick seleccionado ir a paso 4}
		
		\paso{4 Fin de caso de uso}{}
		
	\end{cursoe}

		\paragraph{Observaciones:}
		\begin{flushleft}
		El cliente puede cancelar en todo momento
		\end{flushleft}
	
\end{cu}

\paragraph{Glosario}

\begin{itemize}
	\item \textbf{Casino: } Sistema online del casino
	\item \textbf{Casino Online: } Sistema online del casino
\end{itemize}






%\begin{cu}{Vendiendo pasajes}
%\defln{Este caso de uso expilica la interaccion 
%del vendedor con nuestro sistemas de ventas}
%\actor{vendedor de pasajes}
%\actorsec{vendedor de pasajes secundario}
%\actorsec{sistema online}
%\pre{precondicion}
%\post{postcondicion}
%\begin{cursoe}{}
%\paso{a}{b}
%\end{cursoe}
%\end{cu}
%
%De esta manera escribiremos en caso de uso:
%
%\begin{verbatim}
%\begin{cu}{Vendiendo pasajes}
%\defln{Este caso de uso expilica la interaccion 
%del vendedor con nuestro sistemas de ventas}
%\actor{vendedor de pasajes}
%\actorsec{vendedor de pasajes secundario}
%\actorsec{sistema online}
%\pre{precondicion}
%\post{postcondicion}
%\begin{cursoe}{}
%\paso{a}{b}
%\end{cursoe}
%\end{cu}
%\end{verbatim}

\end{document}



