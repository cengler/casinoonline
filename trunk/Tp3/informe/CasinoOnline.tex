\documentclass[a4paper, 10pt, notitlepage]{article}
 
\usepackage[pdftex]{graphicx}
\usepackage{moreverb} %para importar codigo
%\usepackage{pepotina} %paquete personal para la caratula del DC
\usepackage[spanish,activeacute]{babel}
\usepackage{babel} %paquete de idioma
\usepackage[latin1]{inputenc}
%\usepackage{color}
\usepackage{hyperref}
%\usepackage{caeycaeING}
\usepackage{fancyhdr} %linea sup con comentarios
\usepackage{lscape} %para hoja apaisada
\usepackage{framed} %para crear cajas de texto
\usepackage{lastpage} %ultima pagina

\usepackage{listings}
%\lstset{
%  breaklines=true,          % line wrapping on
%  language=ocl,
%  frame=ltrb,
%  framesep=5pt,
%  basicstyle=\normalsize,
%  keywordstyle=\ttfamily\color{OliveGreen},
%  identifierstyle=\ttfamily\color{CadetBlue}\bfseries,
%  commentstyle=\color{Brown},
%  stringstyle=\ttfamily,
%  showstringspaces=ture
%}
 
\newcommand{\HRule}{\rule{\linewidth}{0.5mm}}

\addtolength{\topmargin}{-50pt} 
\addtolength{\textwidth}{105pt}
\addtolength{\textheight}{120pt}
\addtolength{\oddsidemargin}{-50pt}

\usepackage{fancyhdr} %linea sup con comentarios
\pagestyle{fancy}
\fancyhead[LO]{Ingenieria del Software I}
\fancyhead[C]{}
\fancyhead[RO]{P\'agina \thepage\ de \pageref{LastPage}}
\renewcommand{\headrulewidth}{0.4pt}
\fancyfoot{}

 
\begin{document}

%\setcounter{section}{1} 
 
\begin{titlepage}
 
\begin{center}
 
%% \universidad{Universidad de Buenos Aires}
%%\facultad{Facultad de ciencias exactas y naturales}
%%\departamento{Departamento de Computacion}
%%\materia{Ingenieria del Software I}
%%\resumen{Proyecto casino online}
%%\keys{UML, Objetivos, Agentes, Casos De Uso, Diagrama De Contexto, Modelo Conceptual, OCL, Diagrama de Actividades, FSM}
%%\titulo{Proyecto: Casino Online}
%%\subtitulo{Informe 1: Analisis de Requerimientos y especificaci�n}
%%\grupo{Numero de grupo: 2}
%%\fecha{1er Cuatrimeste 2008}
%%\footspace{1cm}
%%\integrante{Aquino, Isis}{313/05}{isisaquino@yahoo.com.ar}
%%\integrante{Alvarez, Maria de los Angeles}{264/05}{mdelosaalvarez@hotmail.com}
%%\integrante{Engler, Christian Alejandro}{314/05}{caeycae@gmail.com}

 
 
 
 
 
% Upper part of the page
\includegraphics[width=0.5\textwidth]{./logo_uba_p.jpg}\\[1cm]
 
\textsc{\LARGE Universidad de Buenos Aires}\\
\textsc{\LARGE Facultad de ciencias exactas y naturales}\\[1.5cm]
 
\textsc{\Large Departamento de Computacion}\\
\textsc{\Large Ingenieria del Software I}\\[0.5cm]
 
% Title
\HRule \\[0.4cm]
{ \huge \bfseries Proyecto Casino Online}\\[0.4cm]
 
\HRule \\[1.5cm]
 
\begin{center}
	\begin{flushleft} \large
	\emph{Author:}\\
		Isis \textsc{Aquino} $313/05$ \\
		Maria de los Angeles \textsc{Alvarez} $264/05$\\
		Christian Alejandro \textsc{Engler} $314/05$
	\end{flushleft}
\end{center}.\\[2cm]

\begin{center}
	\begin{flushleft} \large
	\emph{Supervisor:} \\
		Sergio \textsc{D'Arrigo}
	\end{flushleft}
\end{center}
 
\vfill
 
% Bottom of the page
{\large \today}
 
\end{center}
 
\end{titlepage}


\tableofcontents

\newpage

\section{Introduccion}

\section{Cambios con respecto al diseno}
\label{sec:CambiosConRespectoAlDiseno}

\section{Implementacion}
\label{sec:Implementacion}

\subsection{Organizacion del Codigo}
\label{sec:OrganizacionDelCodigo}

El la carpeta codigo se encuentra todo el codigo y archivos necesarios para la ejecucion y compilacion de dicho codigo.

\begin{itemize}
	\item /doc - documentacion autogenerada (no  completa)
	\item /etc - Archivos necesarios para ciertos chequeos de codigo
	\item /jar - codigo compilado
	\item /lib - librerias necesarias para la compilacion
	\item /report - varios reporte de calidad de codigo
	\item /src - codigo java
		\subitem /casino - paquete casino
		\subitem /configuracion - archivos de configuracion
			\subsubitem CFGSaldo.xml - configuracion del saldo de los pozos
			\subsubitem fichasValidas.xml - configuracion de fichas validas
			\subsubitem generalConfig.properties - configuracion de minimo pozo feliz y descuento todos ponen
			\subsubitem listaJugadores.xml - lista de jugadores y sus saldos
			\subsubitem mensajeroConfig.properties - filtros de los mensajeros y mailbox
			\subsubitem modificaciones.csv - especifica para cada apuesta de craps cierta accion
			\subsubitem pagos.csv - especifica en que relacion y cuando se deben pagar las apuestas
		\subitem /core - paquete core
		\subitem /craps - paquete craps
		\subitem /interpretador - paquete interpretador
		\subitem /mensajero - paquete mensajero
		\subitem /observerCraps - paquete observador de craps
		\subitem /parser - paquete con los parsers del casino
		\subitem /server - paquete con el iniciador del casino
		\subitem /servicios - paquete servicios
		\subitem log4j.properties - archivo de configuracion de loggeo
	\item /xml - xml de ejemplo
	\item build.xml - archivo ant para la compilacion del proyecto
\end{itemize}

\subsection{Configuracion}
\label{sec:Configuracion}

\subsubsection{Saldos Generales}
\label{sec:SaldosGenerales}
CFGSaldo.xml
\begin{framed}
\begin{verbatim}
<saldos>
	<saldoCasino>1000</saldoCasino>
	<saldoPozoFeliz>1020</saldoPozoFeliz>
</saldos>
\end{verbatim}
\end{framed}

\subsubsection{Fichas Validas}
\label{sec:FichasValidas}
fichasValidas.xml
\begin{framed}
\begin{verbatim}
<list>
	<itemApuesta ficha="5" cantidad="5" />
	<itemApuesta ficha="10" cantidad="10" />
	<itemApuesta ficha="25" cantidad="25" />
	<itemApuesta ficha="50" cantidad="50" />
	<itemApuesta ficha="75" cantidad="75" />
	<itemApuesta ficha="100" cantidad="100" />
</list>
\end{verbatim}
\end{framed}

\subsubsection{Reglas y porcentajes pozos}
\label{sec:ReglasYPorcentajesPozos}
generalConfig.properties
\begin{framed}
\begin{verbatim}
porcentajePozoFeliz=50
minimoMontoPozoFeliz=1000
\end{verbatim}
\end{framed}

\subsubsection{Lista de Jugadores}
\label{sec:ListaDeJugadores}
listaJugadores.xml
\begin{framed}
\begin{verbatim}
<list>
	<jugador nombre="Pablo" saldo="1.234567" vip="false"/>
	<jugador nombre="Miguel" saldo="10000.0" vip="true"/>
	<jugador nombre="Martin" saldo="4648.0" vip="true"/>
	<jugador nombre="Pepe" saldo="100.0" vip="true"/>
	<jugador nombre="Ignacio" saldo="10000.0" vip="false"/>
	<jugador nombre="Andres" saldo="1550.0" vip="false"/>
	<jugador nombre="Juan" saldo="200.1" vip="false"/>
</list>
\end{verbatim}
\end{framed}

\subsubsection{Configuracion de Mensajeros}
\label{sec:ConfiguracionDeMensajeros}
mensajeroConfig.properties
\begin{framed}
\begin{lstlisting}[breaklines=true]
dirMensajeroCraps=D:/casino/messageBox
dirMensajeroCrapsSalida=D:/casino/messageBox
dirMensajeroCasino=D:/casino/messageBox
dirMensajeroAdministracion=D:/casino/messageBox
filtroMensajeroAdministracion=abrirCasino02.*|cerrarCasino02.*|setModo02.*|setJugada02.*|reporteRankingJugadores02.*|reporteEstadoActual02.*
filtroMensajeroCasino=entradaCasino02.*|salidaCasino02.*|estadoCasino02.*
filtroMensajeroCraps=entradaCraps02.*|salidaCraps02.*|tirarCraps02.*|apostarCraps02.*
fittroMensajeroCrapsSalida=NOFILE
\end{lstlisting}
\end{framed}

\subsection{Uso del programa}
\label{sec:UsoDelPrograma}

\section{Testing Funcional}
\label{sec:TestingFuncional}

\subsection{Unidad funcional: entrarCasino}
\label{sec:UnidadFuncionalEntrarCasino}

\subsubsection{Descripcion}
\label{sec:Descripcion}

Este testing evalua una entrada correcta de un cliente al casino, ya sea en modo observador o como un jugador que este registrado del casino. Se valida la terminal virtual, que no se loguee si ya lo estaba antes y que no se pueda loguear un jugador o observador si hay otro con el mismo nombre(lo toma como ya logueado).

\subsubsection{Factores Categorias y Elecciones}
\label{sec:FactoresCategoriasYElecciones}

\subsubsection{Casos de Test}
\label{sec:CasosDeTest}

\subsubsection{Instanciacion}
\label{sec:Instanciacion}


\subsection{Unidad funcional: tirarCraps}
\label{sec:UnidadFuncionalTirarCraps}

\subsubsection{Descripcion}
\label{sec:Descripciontc}

Este testing evalua la validez del tiro de un jugador, asi como el resultado del mismo. el jugador debe estar logueado y dentro de la mesa en la qu quiere tirar, ademas debe ser el tirador de la misma. El tirador puede ganar o perder el tiro segun el resultado, y segun este ultimo como tambien el tipo de jugada que sea se haran los pagos, descuentos y movimiento de pozos corrspondientes.

\subsubsection{Factores Categorias y Elecciones}
\label{sec:FactoresCategoriasYEleccionestc}

\begin{tabular}{||l|l|l|l||} % l = left
\hline
\multicolumn{4}{||c||}{Funcionalidad: EntrarCasino} \\ % c = center
\hline
Factor & Categoria & Eleccion & Clasificaci�n \\
\hline
estadoCasino & Abierto & Si & \\
\hline
 &  & No & [ERROR] \\
\hline
\hline
Ingreso & Modo de ingreso & Observador & \\
\hline
 &  & Jugador & [PROP jug]\\
\hline
\hline
Jugador/Jugadores & Pertenece a & Si & \\
\hline
del Casino &  & No & [IF jug][ERROR]\\
\hline
Cliente & Ya logueado & Si & [ERROR]\\
\hline
 &  & No & \\
\hline
\end{tabular}


\subsubsection{Casos de Test}
\label{sec:CasosDeTesttc}



\begin{tabular}{|l|l|l|l|p{7cm}|} % p{7m} // maximo ancho col -> luego baja
\hline
Casino & Jugador & Modo de & Pertenece a & Resultado \\
Abierto & Logueado & ingreso & jugadores del casino & Esperado \\
\hline
No & - & - & - & ERROR, el casino no esta abierto \\
\hline
Si & Si & - & - & ERROR, el jugador ya esta logueado \\
\hline
Si & No & jugador & No & ERROR, el jugador no pertenece a los jugadores del casino \\
\hline
Si & No & jugador & Si & OK, El jugador a ingresado al casino desde su ID TV en modo Jugador \\
\hline
Si & No & observador & No & OK, el invitado ha ingresado al casino desde su ID TV en modo observador \\
\hline
Si & No & observador & Si & OK, el jugador ha ingresado al casino desde su ID TV en modo observador \\
\hline
\end{tabular}

\subsubsection{Instanciacion}
\label{sec:Instanciaciontc}

\subsubsection{Elecciones}
\subsubsection{Resultados}
\subsubsection{Casos de Test}



\section{Conclusiones}
\label{sec:Conclusiones}

Una de las conclusiones obtenidas de este trabajo es que se hizo muy �til tener un dise�o a la hora de implementar ya que ofrec�a un panorama general del proyecto, sus m�dulos y la relaci�n entre ellos. Adem�s permit�a detectar ciclos de dependencia entre clases que si no hubi�semos tenido un dise�o detallado de lo que debe ser implementado hubiera sido muy dificil detectarlo y, por supuesto hubiera complicado el testeo y futuras modificaciones al dise�o y/o implementaci�n.
Los diagramas de secuencias nos fueron d suma utilidad en la implementaci�n,que tuvo ciertos aspectos sencillos en cuanto a complejidad de los m�todos y las clases implementadas las cuales terminaron siendo simples (cada m�todo por si mismo) y en consecuencia f�ciles de entender y seguir. 

\end{document}
