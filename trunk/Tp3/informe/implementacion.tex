\section{Implementacion}
\label{sec:Implementacion}

\subsection{Organizacion del Codigo}
\label{sec:OrganizacionDelCodigo}

El la carpeta codigo se encuentra todo el codigo y archivos necesarios para la ejecucion y compilacion de dicho codigo.

\begin{itemize}
	\item /doc - documentacion autogenerada (no  completa)
	\item /etc - Archivos necesarios para ciertos chequeos de codigo
	\item /jar - codigo compilado
	\item /lib - librerias necesarias para la compilacion
	\item /report - varios reporte de calidad de codigo
	\item /src - codigo java
		\subitem /casino - paquete casino
		\subitem /configuracion - archivos de configuracion
			\subsubitem CFGSaldo.xml - configuracion del saldo de los pozos
			\subsubitem fichasValidas.xml - configuracion de fichas validas
			\subsubitem generalConfig.properties - configuracion de minimo pozo feliz y descuento todos ponen
			\subsubitem listaJugadores.xml - lista de jugadores y sus saldos
			\subsubitem mensajeroConfig.properties - filtros de los mensajeros y mailbox
			\subsubitem modificaciones.csv - especifica para cada apuesta de craps cierta accion
			\subsubitem pagos.csv - especifica en que relacion y cuando se deben pagar las apuestas
		\subitem /core - paquete core
		\subitem /craps - paquete craps
		\subitem /interpretador - paquete interpretador
		\subitem /mensajero - paquete mensajero
		\subitem /observerCraps - paquete observador de craps
		\subitem /parser - paquete con los parsers del casino
		\subitem /server - paquete con el iniciador del casino
		\subitem /servicios - paquete servicios
		\subitem log4j.properties - archivo de configuracion de loggeo
	\item /xml - xml de ejemplo
	\item build.xml - archivo ant para la compilacion del proyecto
\end{itemize}

\newpage

\subsection{Configuracion}
\label{sec:Configuracion}

\subsubsection{Saldos Generales}
\label{sec:SaldosGenerales}

\large{\textit{CFGSalgo.xml}}\\[0.5cm]
En este archivo se configuran los saldos del casino, tanto el saldo general como el saldo de los pozos de premios.\\
Este archivo se carga en el momento de abrir el casino y se actualiza en el momento de cerrarlo (con los valores actualizados luego de los movimientos de saldo).

\begin{framed}
\begin{verbatim}
<saldos>
	<saldoCasino>1000</saldoCasino>
	<saldoPozoFeliz>1020</saldoPozoFeliz>
</saldos>
\end{verbatim}
\end{framed}


\subsubsection{Fichas Validas}
\label{sec:FichasValidas}

\large{\textit{fichasValidas.xml}}\\[0.5cm]
En este archivos se configuran las fichas validas del casino, el primer valor representa el id de la ficha y el segundo el valor de la ficha. Los clientes apuestan por id de ficha (esto para tener indireccion con el valor real de las fichas).

\begin{framed}
\begin{verbatim}
<list>
	<valorFicha ficha="5" valor="5" />
	<valorFicha ficha="10" valor="10" />
	<valorFicha ficha="25" valor="25" />
	<valorFicha ficha="50" valor="50" />
	<valorFicha ficha="75" valor="75" />
	<valorFicha ficha="100" valor="100" />
</list>
\end{verbatim}
\end{framed}

\subsubsection{Reglas y porcentajes pozos}
\label{sec:ReglasYPorcentajesPozos}

\large{\textit{generalConfig.properties}}\\[0.5cm]
Configuracon de propiedades generales del casino, de esta manera se puede configurar el porecentaje de descuento del pozo feliz y el monto minimo para que se entregue el este pozo.

\begin{framed}
\begin{verbatim}
porcentajePozoFeliz=50
minimoMontoPozoFeliz=1000
\end{verbatim}
\end{framed}

\newpage

\subsubsection{Lista de Jugadores}
\label{sec:ListaDeJugadores}

\large{\textit{listaJugadores.xml}}\\[0.5cm]
Lista de los jugadores registrados con sus saldos y propiedades. Se carga en el momento de abrir el casino y se persiste en el momento de cerrarlo. Las unicas modificaciones que hace el casino sobre esta archivo es el monto de los saldos que poseen los jugadores.

\begin{framed}
\begin{verbatim}
<list>
	<jugador nombre="Pablo" saldo="1.234567" vip="false"/>
	<jugador nombre="Miguel" saldo="10000.0" vip="true"/>
	<jugador nombre="Martin" saldo="4648.0" vip="true"/>
	<jugador nombre="Pepe" saldo="100.0" vip="true"/>
	<jugador nombre="Ignacio" saldo="10000.0" vip="false"/>
	<jugador nombre="Andres" saldo="1550.0" vip="false"/>
	<jugador nombre="Juan" saldo="200.1" vip="false"/>
</list>
\end{verbatim}
\end{framed}

\subsubsection{Configuracion de Mensajeros}
\label{sec:ConfiguracionDeMensajeros}

\large{\textit{mensajeroConfig.properties}}\\[0.5cm]
Configuracion de los mensajeros. Cada puerto de acceso al sistema puede trabajar en una carpeta diferente.\\
Por otro lado cada mensajero debe encargarse de los archivos que le son propios, los filtros que estan configurados en este archivo, son expresiones regulares que deben machear con los nombres de los archivos a procesar.\\
Es importente que las expresiones regulares no macheen con las respuestas del casino, pues los archivos de salida se entregan en la misma carpeta.

\begin{framed}
\begin{lstlisting}[breaklines=true]
dirMensajeroCraps=D:/casino/messageBox
dirMensajeroCrapsSalida=D:/casino/messageBox
dirMensajeroCasino=D:/casino/messageBox
dirMensajeroAdministracion=D:/casino/messageBox
filtroMensajeroAdministracion=abrirCasino02.*|cerrarCasino02.*|setModo02.*|setJugada02.*|reporteRankingJugadores02.*|reporteEstadoActual02.*
filtroMensajeroCasino=entradaCasino02.*|salidaCasino02.*|estadoCasino02.*
filtroMensajeroCraps=entradaCraps02.*|salidaCraps02.*|tirarCraps02.*|apostarCraps02.*
fittroMensajeroCrapsSalida=NOFILE
\end{lstlisting}
\end{framed}

\newpage

\subsection{Uso del programa}
\label{sec:UsoDelPrograma}

\subsubsection{compilacion}
\label{sec:compilacion}

Utilizamos la heramienta \textsl{ant}\footnote{Ant es una herramienta Open-Source utilizada en la compilaci�n y creaci�n de programas Java. http://ant.apache.org/} para realizar la compilacion del proyecto.\\

\begin{verbatim}
Main targets:

 all      Realiza todo.
 check    Genera reporte de violaciones de convenciones de codigo.
 clear    Borra para un recompilacion completa
 dot      Genera grafo de dependencias.
 javadoc  Genera javadoc.
 jdepend  Genera reporte de dependencias
 make     Compila el proyecto
 makedir  Crea los directorios
 run      Ejecuta el casino.
 zip      Crea el jar
Default target: zip
\end{verbatim}

Para compilar el proyecto sera necesario situarse en /codigo y ejecutar el comando \verb|ant|. Que ejecuta en target \textit{zip} (esta target incluye \textit{clear}, \textit{makedir}, \textit{make} y \textit{zip}).\\
De esta manera en el directorio /codigo/jar se copiaran todos los archivos necesarios para la ejecucion.

\subsubsection{ejecucion}
\label{sec:ejecucion}

\large{\textit{Por linea de comandos.}}

\begin{verbatim}
\casino\codigo\jar\java -cp ..\lib\log4j.jar;..\lib\opencsv-1.8.jar;
       ..\lib\xstream-1.3.jar;casino.jar server.Start
\end{verbatim}

\vskip1cm

\large{\textit{Con ant.}}
\begin{verbatim}
\casino\codigo\jar\ant run
\end{verbatim}

\vskip1cm

\large{\textit{RECUERDE:}}


\vskip0.3cm

La configuracion del directorio de trabajo de los mensajeros se configura en: \\ \verb|/configuration/mensajeroConfig.properties|\\
En el caso de correr el trabajo desde el directorio jar el archivos de encontrar� en: \\
\verb|/casino/codigo/jar/configuration/mensajeroConfig.properties|\\

\newpage

\subsection{Loggeo}
\label{sec:Loggeo}

Para administrar el loggeo de la aplicacion utilizamos log4j\footnote{log4j es una biblioteca open source que permite a los desarrolladores de software elegir la salida y el nivel de granularidad de los mensajes o ``logs'' (logging) a tiempo de ejecuci�n.} \\
De esta manera la configuraci�n de salida y granularidad de los mensajes es realizada a tiempo de ejecuci�n mediante el uso de un archivo: \verb|log4j.properties|\\

Las dos ultimas lineas contienen el nivel de loggeo que se desea obtener. Este pude ser TRACE, DEBUG, INFO, ERROR y FATAL.\\
Tambien se configura en este archivo la salida por la cual deberan mostrarse los mensajes de loggeo.
El loggeo general del sistema esta direccionado a la cosola de ejecucion y a un archivo de log \verb|server.log|\\
El loggeo de mensajeria esta dirreccionada tambien a la consola de ejecicion y al archivo \verb|messages.log| (Este logger sirve para ver los xml que van llegando y saliendo del servidor).

\begin{verbatim}
log4j.rootLogger=info, stdout, logfile 
log4j.logger.MessageParser=debug, stdout, msgfile
\end{verbatim}

\subsection{Reportes}
\label{sec:Reportes}

Hay varios reportes generados a partir del codigo\footnote{los reportes se encuentran en la carpeta /codigo/report, si no se encuentran o no se encuentra la carpeta, se debera ejecutar el commando \textsc{ant all}}. El mas interesante es un grafo de dependencias que se genera a partir del codigo cuando se ejecuta el commando \verb|ant dot|. Es una forma grafica de estudiar las dependencias del codigo.

\vskip1cm

Tambien estan dos reportes mas, uno generado con el jdepend que lista las dependencias entre paquetes y uno generado por el checkstyle para evaluar el cumplimiento de convenciones de codificacion.

