\section{Implementacion}
\label{sec:Implementacion}

\subsection{Organizacion del Codigo}
\label{sec:OrganizacionDelCodigo}

El la carpeta codigo se encuentra todo el codigo y archivos necesarios para la ejecucion y compilacion de dicho codigo.

\begin{itemize}
	\item /doc - documentacion autogenerada (no  completa)
	\item /etc - Archivos necesarios para ciertos chequeos de codigo
	\item /jar - codigo compilado
	\item /lib - librerias necesarias para la compilacion
	\item /report - varios reporte de calidad de codigo
	\item /src - codigo java
		\subitem /casino - paquete casino
		\subitem /configuracion - archivos de configuracion
			\subsubitem CFGSaldo.xml - configuracion del saldo de los pozos
			\subsubitem fichasValidas.xml - configuracion de fichas validas
			\subsubitem generalConfig.properties - configuracion de minimo pozo feliz y descuento todos ponen
			\subsubitem listaJugadores.xml - lista de jugadores y sus saldos
			\subsubitem mensajeroConfig.properties - filtros de los mensajeros y mailbox
			\subsubitem modificaciones.csv - especifica para cada apuesta de craps cierta accion
			\subsubitem pagos.csv - especifica en que relacion y cuando se deben pagar las apuestas
		\subitem /core - paquete core
		\subitem /craps - paquete craps
		\subitem /interpretador - paquete interpretador
		\subitem /mensajero - paquete mensajero
		\subitem /observerCraps - paquete observador de craps
		\subitem /parser - paquete con los parsers del casino
		\subitem /server - paquete con el iniciador del casino
		\subitem /servicios - paquete servicios
		\subitem log4j.properties - archivo de configuracion de loggeo
	\item /xml - xml de ejemplo
	\item build.xml - archivo ant para la compilacion del proyecto
\end{itemize}

\subsection{Configuracion}
\label{sec:Configuracion}

\subsubsection{Saldos Generales}
\label{sec:SaldosGenerales}
CFGSaldo.xml
\begin{framed}
\begin{verbatim}
<saldos>
	<saldoCasino>1000</saldoCasino>
	<saldoPozoFeliz>1020</saldoPozoFeliz>
</saldos>
\end{verbatim}
\end{framed}

\subsubsection{Fichas Validas}
\label{sec:FichasValidas}
fichasValidas.xml
\begin{framed}
\begin{verbatim}
<list>
	<itemApuesta ficha="5" cantidad="5" />
	<itemApuesta ficha="10" cantidad="10" />
	<itemApuesta ficha="25" cantidad="25" />
	<itemApuesta ficha="50" cantidad="50" />
	<itemApuesta ficha="75" cantidad="75" />
	<itemApuesta ficha="100" cantidad="100" />
</list>
\end{verbatim}
\end{framed}

\subsubsection{Reglas y porcentajes pozos}
\label{sec:ReglasYPorcentajesPozos}
generalConfig.properties
\begin{framed}
\begin{verbatim}
porcentajePozoFeliz=50
minimoMontoPozoFeliz=1000
\end{verbatim}
\end{framed}

\subsubsection{Lista de Jugadores}
\label{sec:ListaDeJugadores}
listaJugadores.xml
\begin{framed}
\begin{verbatim}
<list>
	<jugador nombre="Pablo" saldo="1.234567" vip="false"/>
	<jugador nombre="Miguel" saldo="10000.0" vip="true"/>
	<jugador nombre="Martin" saldo="4648.0" vip="true"/>
	<jugador nombre="Pepe" saldo="100.0" vip="true"/>
	<jugador nombre="Ignacio" saldo="10000.0" vip="false"/>
	<jugador nombre="Andres" saldo="1550.0" vip="false"/>
	<jugador nombre="Juan" saldo="200.1" vip="false"/>
</list>
\end{verbatim}
\end{framed}

\subsubsection{Configuracion de Mensajeros}
\label{sec:ConfiguracionDeMensajeros}
mensajeroConfig.properties
\begin{framed}
\begin{lstlisting}[breaklines=true]
dirMensajeroCraps=D:/casino/messageBox
dirMensajeroCrapsSalida=D:/casino/messageBox
dirMensajeroCasino=D:/casino/messageBox
dirMensajeroAdministracion=D:/casino/messageBox
filtroMensajeroAdministracion=abrirCasino02.*|cerrarCasino02.*|setModo02.*|setJugada02.*|reporteRankingJugadores02.*|reporteEstadoActual02.*
filtroMensajeroCasino=entradaCasino02.*|salidaCasino02.*|estadoCasino02.*
filtroMensajeroCraps=entradaCraps02.*|salidaCraps02.*|tirarCraps02.*|apostarCraps02.*
fittroMensajeroCrapsSalida=NOFILE
\end{lstlisting}
\end{framed}

\subsection{Uso del programa}
\label{sec:UsoDelPrograma}