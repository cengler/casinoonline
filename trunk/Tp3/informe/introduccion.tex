\section{Introduccion}

\subsection{Objetivo del documento}

El objetivo de este documento es presentar un analisis detallado del proceso de implementacion a partir del dise�o y de la presentacion del testing funcional realizado sobre las 2 funcionalidades pedidas.
Con respecto a la implementacion, se presentan algunas decisiones que tuvimos que tomar que provocaron cambios con respecto al dise�o presentado, como asi la explicitacion del nivel de independencia alcanzado. Ademas incluimos una detallada explicaci�n de las librer�as y paquetes utilizados en el codigo, como asi tambi�n el protocolo de administracion del casinoOnLine.
En el caso del testing funcional, las 2 funcionalidades requeridas est�n detalladas y explicadas en lenguaje natural, con algunos detalles de las cosas que tuvimos en cuenta o que fueron descartadas del test o que asumimos para realizarlo.
A continuaci�n se prensentan algunos documentos que se podr�an consultar y en los que hemos basado este tercer informe.

\subsection{Documentos relacionados}
	
	Detallamos aqui algunos documentos relacionados:
	
\begin{itemize}
  \item Dise�o CasinoOn Line
	\item Arquitectura Conceptual de la Aplicaci�n y Protocolo de comunicaci�n del Casino OnLine y respectivas actualizaciones.
	\item Modificaciones al protocolo, 17-06-2008
	\item Reglas Craps
\end{itemize}

Encontrar� mas informaci�n y bibliografia en la seccion Referencias.
	
\subsection{Organizaci�n del informe}

\textbf{El Informe esta organizado de la siguiente manera:}

\begin{description}
	\item[Cambios con respecto al dise�o] Decisiones tomadas a la hora de implementar que han provocado cambios con respecto al dise�o presentado con anterioridad. Incluye las respectivas justificaciones que validadan estas modificacionen realizadas. Se detalla el nivel de dependencia alcanzado y se lo podr� observar en un grafico representativo.
	\item[Protocolo Administracion] Descripcion de los archivos xml utilizados para la manipulaci�n del casino. 
	\item[Implementacion] Explicacion de detalles implemantativos, descripcion de funcionalidades, utilidad de archivos, configuraciones iniciales. Incluimos algunas sugerencias para la compilacion y ejecucion de la aplicacion.
	\item[Testing funcional] Descripcion del testing funcional sobre la funcionalidad. Tabla de factores, categorias y elecciones implicadas en la funcionalidad a testear. Tabla de casos de test con sus respectivos resultados esperados. Datos de prueba utilizados al testear la funcionalidad y descripcion del proceso de testeo al que fue sometida la aplicacion como tambien el resultado de los mismos.
	\item[Conclusiones] Conclusiones obtenidas de esta tercera parte del trabajo.
	\item[Referencias] Referencias bibliograficas que se pueden consultar.
\end{description}


